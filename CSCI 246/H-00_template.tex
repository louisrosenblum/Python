\documentclass{article}
\usepackage{fasy-hw}

\author{Anthony Louis Rosenblum}
\problem{1}
\collab{Mr. Macky (2014-2015), BBC - Horizon - 1996 - Fermat's Last Theorem}


\begin{document}

\emph Question 1: My photo in D2L has been updated to be a clearly identifiable photo

\emph Question 3: 


\section(3A) Algorithms: A process which performs a desired task in a set manner. I first learned about algorithms in middle school when all of my friends were very much into Rubick's Cubes (I however was not). I am aware of sorting algorithms but I do have much experience writing or understanding algorithms thus far.
\section(3B) Data Structures: A system to organize data. Through 112, 127, and 132 I am aware of lists, linked lists, trees, objects, and other ways to organize data. 
\section(3C) Graphs: I have been familiar with graphs for many year now. Between middle and high school math as well as MATLAB experience in courses like circuits and even my Calculus classes.
\section(3D) Binomial Coefficients: I'm not sure exactly what this is referring to but I believe it has something to do with pascal's triangle. Expanding the product of two or more polynomials by finding all coefficients.
\section(3E) Proof by Counter-example. If I want to disprove A, I can start by assuming A is true, and then finding an example where it is not.
\section(3F) Proof by Induction: I never really understood the difference between induction and deduction. I believe induction has to do with inferring certain things, I would like to become more familiar with this concept.
\section(3G) Recursion in code. I am familiar with this concept from both 127 and 132. The idea of this is creating a function (or algorithm) and then performing such function over and over again on each intermediate output of the function until the base case is reached.
\section(3H) Recurrence Relations: I have never heard this phrase before. I am interested in learning what it means.
\section(3I) The Four Color Theorem: A solution to a problem wondering how many different colors you would need in order to color a map where every entity is a different color than its direct neighbor. My question about four colors being the solution if what if there is a nation that has five or six other nations directly bordering it.



\emph Question 4: I have reviewed all properties of real numbers in Appendix A.

\emph Question 5: There are distinct integers m and n such that 1/m + 1/n is an integer.

		To prove this case I choose m = 1 and n = -1. Now this expression becomes 1/1 + 1/-1 which becomes 1 - 1 which becomes 0.
		Because 0 is an integer, I have proven this statement true for at least one distinct set of integers.
		
\emph Question 6:

		There are real numbers a and b such that root(a+b) = root(a) + root(b).
		
		I select a = 0 and b = 1. The sum of which is 1. The square root of 1 is equal to the square root of 0 plus the square root of 1.
		I have proven this true for at least one case.
		
\emph Question 7: Set A is equal to set C. None of the other sets are equal to each other.

\emph Question 8: T is not a function because ther can be more that one co-domain value for every domain value. Take for example x=0. Here y=1 and also y=-1 satisifes the equation, therefore T is not a function.

\emph 
\collab(Mr. Macky (2014-2015), BBC - Horizon - 1996 - Fermat's Last Theorem)

Question 9: I first learned about Andrew Wiles in 10th grade. I was enrolled in honors pre-calculus with Mr. Macky, who had a very hot and cold reputation among students. One day in class Mr. Macky showed us a documentary on Andrew Wiles and his quest to prove Fermat's last theorem.
				  That theorem being that there is no set of numbers for which a cubed plus b cubed equals c cubed. Supposedly Fermat fit his proof for this theorem either in a margin or in half a piece of paper, that was subsequently lost over the years.
				  It took hundreds of years for anyone to prove Fermat's last theorem, and the amount of work that Andrew Wiles put into it and the length to which his proof goes raises the question of whether Fermat actually proved his theorem or his proof was not complete.
				  Wiles dedicated so much time to this proof, it was honestly his life for several years. I greatly admire Andrew Wiles and am happy he was able to get the recognition he deserves. I think he is an inspiration to all computer scientists knowing that anything can be accomplished with enough effort, and enough proof.
				  Wiles had to dive into so many different realms of knowledge in order to prove that theorem, he embodies what dedication to human knowledge and advancement truly means.
		
		
	



\end{document}
